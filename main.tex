%%%%%%%%%%%%%%%%%%%%%%%%%%%%%%%%%%%%%%%%%%%%%%%%%%%%%%%%%%%%%%%
%
% Welcome to Overleaf --- just edit your LaTeX on the left,
% and we'll compile it for you on the right. If you open the
% 'Share' menu, you can invite other users to edit at the same
% time. See www.overleaf.com/learn for more info. Enjoy!
%
%%%%%%%%%%%%%%%%%%%%%%%%%%%%%%%%%%%%%%%%%%%%%%%%%%%%%%%%%%%%%%%
\documentclass{article}
\begin{document}

Problem 1:\\
a.\\
$\mid\Psi_1\rangle=\frac{1}{\sqrt{3}}\mid+\rangle \;+ \; i\frac{\sqrt{2}}{\sqrt{3}}\mid-\rangle$\\
$\langle\Psi_1\mid=\frac{1}{\sqrt{3}}\mid+\rangle \;+ \; i\frac{\sqrt{2}}{\sqrt{3}}\mid-\rangle$\\
$\langle\Psi_1\mid\Psi_1\rangle=(\frac{1}{\sqrt{3}}\mid+\rangle \;+ \; i\frac{\sqrt{2}}{\sqrt{3}}\mid-\rangle)(\frac{1}{\sqrt{3}}\mid+\rangle \;+ \; i\frac{\sqrt{2}}{\sqrt{3}}\mid-\rangle)$\\
$\langle\Psi_1\mid\Psi_1\rangle=(\frac{1}{\sqrt{3}}*\frac{1}{\sqrt{3}}\langle+\mid+\rangle)+(i\frac{\sqrt{2}}{\sqrt{3}}\langle+\mid-\rangle)\;+\;(i\frac{\sqrt{2}}{\sqrt{3}}\langle+\mid-\rangle)(\frac{1}{\sqrt{3}}*\frac{1}{\sqrt{3}}\langle-\mid-\rangle)$\\
$\langle\Psi_1\mid\psi_1\rangle=\frac{1}{3}(1)\;+\;\frac{2}{3}(0)\;+\;\frac{2}{3}(0)\;+\;\frac{2}{3}(1)$\\
$\langle\Psi_1\mid\Psi_1\rangle=\frac{1}{3}\;+\;\frac{2}{3}=1$\\
$\langle\Phi_1\mid\Phi_1\rangle=1$\\

b.\\
$\mid\Psi_1\rangle=\frac{1}{\sqrt{3}}\mid+\rangle \;+ i\frac{\sqrt{2}}{\sqrt{3}}\mid-\rangle$\\



a.\\
$\mid\Psi_2\rangle=\frac{1}{\sqrt{5}}\mid+\rangle \;-\; \frac{\sqrt{2}}{\sqrt{5}}\mid-\rangle$\\
$\langle\Psi_2\mid=\frac{1}{\sqrt{5}}\langle+\mid \;-\; \frac{\sqrt{2}}{\sqrt{5}}\langle-\mid$\\
$\langle\Psi_2\mid\Psi_2\rangle=(\frac{1}{\sqrt{5}}\mid+\rangle \;+ \frac{\sqrt{2}}{\sqrt{5}}\mid-\rangle)(\frac{1}{\sqrt{5}}\langle+\mid \;+ \frac{\sqrt{2}}{\sqrt{5}}\langle-\mid)$\\
$\langle\Psi_2\mid\Psi_2\rangle=(\frac{1}{\sqrt{5}}*\frac{1}{\sqrt{5}}\langle+\mid+\rangle)+(\frac{1}{\sqrt{5}}*\frac{\sqrt{2}}{\sqrt{5}}\langle+\mid-\rangle)+(\frac{1}{\sqrt{5}}*\frac{\sqrt{2}}{\sqrt{5}}\langle+\mid-\rangle)(\frac{\sqrt{2}}{\sqrt{5}}*\frac{\sqrt{2}}{\sqrt{5}}\langle-\mid-\rangle)$\\
$\langle\Psi_2\mid\Psi_2\rangle=\frac{1}{5}(1)+\frac{2}{3}(0)+\frac{2}{3}(0)+\frac{2}{5}(1)$\\
$\langle\Psi_2\mid\Psi_2\rangle=\frac{1}{5}+\frac{2}{5}=1$\\
$\langle\Phi_2\mid\Phi_2\rangle=\frac{3}{5}$\\

b.\\
$\mid\Psi_2\rangle=\frac{1}{\sqrt{5}}\mid+\rangle \;-\; \frac{\sqrt{2}}{\sqrt{5}}\mid-\rangle$\\

a.\\
$\mid\Psi_3\rangle=\frac{1}{\sqrt{2}}\mid+\rangle \;+\; \frac{e^{i\pi/4}}{\sqrt{2}}\mid-\rangle$\\
$\langle\Psi_3\mid=\frac{1}{\sqrt{2}}\langle+\mid \;+\; \frac{e^{i\pi/4}}{\sqrt{2}}\langle-\mid$\\
$\langle\Psi_3\mid\Psi_3\rangle=(\frac{1}{\sqrt{2}}\langle+\mid \;+\; \frac{e^{i\pi/4}}{\sqrt{2}}\langle-\mid)(\frac{1}{\sqrt{2}}\mid+\rangle \;+\; \frac{e^{i\pi/4}}{\sqrt{2}}\mid-\rangle)$\\
$\langle\Psi_3\mid\Psi_3\rangle=\frac{1}{\sqrt{2}}*\frac{1}{\sqrt{2}}\langle+\mid+\rangle \;+\; \frac{1}{\sqrt{2}}*\frac{e^{i\pi/4}}{\sqrt{2}}\langle-\mid+\rangle \;+\; \frac{e^{i\pi/4}}{\sqrt{2}}*\frac{1}{\sqrt{2}}\langle-\mid+\rangle \;+\; \frac{e^{i\pi/4}}{\sqrt{2}}*\frac{e^{i\pi/4}}{\sqrt{2}}\langle-\mid-\rangle$\\
$\langle\Psi_3\mid\Psi_3\rangle=\frac{1}{2}(1)\;+\;(0)+(0)+i(1)$\\
$\langle\Psi_3\mid\Psi_3\rangle=\frac{1}{2}+i$



\end{document}
